\documentclass[12pt, a4paper]{article}
\usepackage{multicol}
\usepackage{etoolbox}
\usepackage{relsize}
\apptocmd{\sloppy}{\hbadness 10000\relax}{}{}

\title{Workweek}
\author{Unknown}
\begin{document}
    \pagenumbering{gobble}
    \maketitle

    \section{Background}
    We are used to go to work, do our eight hours, and then go home. In late 18th century,the time spent working per day could reach as high as 16 hours, and was seldom shorter than 10. The long days was a result of the industrial revolution. In the early 20th centory United States, the working were still stretching s long as 10 hours and 12 in the steel industry, and yet the the pay did not reach minimum deemed necessary for a decent life.\cite{uslabor} 
    The eight-hour work day is a product of the regulation made by the social movement, and the contemporary workday is sprung from this movement.\cite{theeighthourday}

    However, times have changed, and the movements regulating the working hours to reform the days of labour in the heavy industry were doing their part on the industrial era---these industries are not any longer the ones dominating the market.
    The age of information has radically shifted the power to the companies that are povited around this information and data. The shift has carries with it a change in the way we work, what tasks are at stake, and our approach to our everyday jobs.

    \begin{multicols}{2}
        [ 
        \section{Inefficiency}
        The amount of work we perform on an everyday basis, on average, could, to say the least, lack efficiency.
        \newpage
        \subsection{How Efficient is the Actual Work}
Studies have shown that white collar workers in the UK does not feel productive throughout the workday 
        ]

        In a servey done in the UK, people that worked in differennt offices were asked to appreciate the time they were putting down actual work when being at the office. While many think that there is work taking place when being at the office, the time spent on the actual task, is not close to what is expected.\cite{Ltd2021Nov}
        A few short breakes beside breakfast, lunch, and an afternoon tea could be what is making the productive time go to waste, and it would be a fitting though too. Perhaps a good hour is totally wasted when stretching the muscles and taking a short break to put your eyes on something else than then computer monitor.
        This is, however, far from the truth. The participants in the servey did ont feel like they took an hour away per day, not even two. The average answer to how much time the office workers felt they were actually being productive was merely 2 hours and 53 minutes on a working day.

        Another question that the survey included was `Do you consider yourself to be productive throughout the entire working day?'. 79\% claimed that this was not the case, and the test, 21\%, said they were.\cite{Ltd2021Nov}

    \end{multicols}
    \subsection{Email and Phones}
    \begin{multicols}{2}
        The more time we have, the more things we could potentially get done. However, this is not the case when shortening the working hours per week.
        By cutting down the time spend in the office, we could potentially create a more productive one. According to a study made by Adobe, the 400 US-based white collar workers were more or less addicted emails.\cite{adobemail} A staggering number of six hours was were how much the workers admittedly spent reading and answering mail per day. Many of the office workers reported that handling email was fatiguing, indicating that the large part of the day were spent doing something that lowered both focus and productivity. 

        Another study found that every day, on averge, we swipe and click our phone 2,617 times and an active usage or 145 minutes, per day.\cite{dscoutphone} The heavy users could reach as high as 5,427 touches and 225 average spent minutes on the phone per day.
    \end{multicols}
        
    \noindent\fbox{%
        \parbox{\dimexpr\linewidth-2\fboxsep-2\fboxrule}{%
    A typical day of an employee comprises of long meetings, unplanned interruptions, unnecessary consensus-seeking, and more.\cite{harvard6hour}
        }%
    }
    \subsection{Boost productivity by cutting hours}
        The more time we have, the more things we could potentially get done. However, this is not the case when shortening the working hours per week.

    Steve Glaveski conducted a two-week, six-hour workday experiment with his team a Collective Campus.\cite{harvard6hour} By cutting the day, the team was forced to make more effective planning of the day. The teammembers were positively affected by the change, reporting elevated mental state and more time for their families. By prioritizing high-value tasks, cutting down time spent in meetings, turning off notifications, and stop frequently checking your email, the waste of time was already an improvement.
    To let the employees get into a work flow where minimal disturbance was applied made the work more effective and also eliviated stress. 

    To decrease the stress at work, U.S. Army civilian employees were taken off their emailing for a week. The action made them feel less stressed, more in control and more productive.\cite{jettisoningworkemail}

    \subsubsection{Trials in the health care sector}
    In Sweden, at Svartedalens nursing home, trials have been conducted where the nurses switched from an eight-hour work day to six hours.\cite{swedennurses} The wage sayed the same. The same application was made on Gothenburg's Sahlgrenska University Hospital, as well as two hospital departments in Umeå. A balance between work and life was improved upon and the nurses energy on work was prominent.

    \subsubsection{Toyota, Sweden}
    In Gothenburg, Sweden, Toyota has shortened the working hours from eight to six for more than a decade.\cite{swedennurses} With full pay, the staff feel that they are happied and more efficient. Profits have risen by 25\%.

    \subsubsection{The startup Brath}
    In Stockholm, Sweden, a startup company named Brath lead by Maria Bråth, have done the same transition on working hours.\cite{brath} Brath is profitable and claims that they are getting more done than their competitors that do eight-hour workdays.

    \subsubsection{Microsoft, Japan}
    Microsoft, Japan made had a different approach to cutting down the working hour in a week. They shortened the week from five to four days.\cite{microsoft} The trial was held under a six-week period, and bore with it some positive impact. Stepping down from 40 to 32 hours work per week, the productivity jumped by 20\% and the employees showed a 24\% improvement in their work-life balance and the stress levels decreased with 7\%.

    \subsubsection{Iceland}
I huge experiment was conducted on Iceland where about 2500 people, or 1\% of Iceland's population attended.\cite{iceland} It stretched from 2015--2019, and the different worktitles stretched from office workers, to health care staff, people working 9-5, to people working shift. The experiment involved lowering the hours worked per week from 40 to 35- to 36 and the wage persisted. The experiment was a success, where the productivity remained the same or even rose in some occations. The workers were happier, healthier, and their well-being was improved; their stresslevels decreased, and their life- and work-balance also improved.

    %\begin{multicols}{2}
    %\bibliographystyle{amsplain}
    %\bibliography{workweek}
    %\end{multicols}
    \section*{References}
    \begin{multicols}{2}
        \small
        \renewcommand{\refname}{\vspace{-\baselineskip}\vspace{-1.2mm}}
        \bibliographystyle{abbrv}
        \bibliography{workweek}
    \end{multicols}
  
\end{document}

